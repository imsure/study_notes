%%%%%%%%%%%%%%%%%%%%%%%%%%%%%%%%%%%%%%%%%%%%%%%%%%%%%%%%%%%%%%%%%%%%%%%%%%%
% This is LaTeX file for Solutions of Homework Assignment 4
% Author: Shuo Yang
%%%%%%%%%%%%%%%%%%%%%%%%%%%%%%%%%%%%%%%%%%%%%%%%%%%%%%%%%%%%%%%%%%%%%%%%%%%

\documentclass[11pt]{article}
\usepackage{amsmath,amssymb,epsfig,graphics,hyperref,amsthm}

\hypersetup{colorlinks=true}

\setlength{\textwidth}{7in}
\setlength{\topmargin}{-0.575in}
\setlength{\textheight}{9.25in}
\setlength{\oddsidemargin}{-.25in}
\setlength{\evensidemargin}{-.25in}

\reversemarginpar
\setlength{\marginparsep}{-15mm}

\newcommand{\rmv}[1]{}
\newcommand{\bemph}[1]{{\bfseries\itshape#1}}
\newcommand{\N}{\mathbb{N}}
\newcommand{\Z}{\mathbb{Z}}
\newcommand{\imply}{\to}
\newcommand{\bic}{\leftrightarrow}

% Some user defined strings for the homework assignment
%
\def\CourseCode{Math4CS Notes}
\def\DateHandedOut{Dec, 2014}
\def\DateDue{Dec 28}
\def\Author{Shuo Yang}

\begin{document}

\noindent

\CourseCode \hfill \DateHandedOut

\begin{center}
Math4CS\\
Date: \DateDue\\
Student: \Author\\
\end{center}

% A horizontal split line
\hrule\smallskip

\section{Induction}

% Enumerate through all questions.
\begin{enumerate}

\item \textbf{Courtyard Tiling}\\

  The problem is to tile a courtyard with dimensions $2^n \times
  2^n$. We are required to install a statue of a wealthy donor in one
  of the central square, and only special L-shaped tiles can be
  used. We need to prove this is feasible.
  
  \textbf{Theorem} For all $n \geq 0$ there exists a tiling of a $2^n
  \times 2^n$ courtyard with the donor in a central square. 
  \begin{proof}
    Prove by induction. Let $P(n)$ be the proposition that there
    exists a tiling of a $2^n \times 2^n$ courtyard with the donor
    placed in any location.

  \emph{Base case}: $P(0)$ is true because the donor fills the whole
  courtyard.

  \emph{Inductive step}: Suppose $P(n)$ is true, we need to prove that
  $P(n) \rightarrow P(n+1)$. A $2^{n+1} \times 2^{n+1}$ courtyard
  consists of four $2^n \times 2^n$ quadrants, each of them can be tiled
  with the donor placed in any location. Let the donor be in one of
  the four central squares, and the remaining three central squares
  can fit a L-shaped tile. Now we can tile each of the four quadrants
  by the induction hypothesis. This proves that $P(n) \rightarrow
  P(n+1)$. The theorem follows as a special case.
  \end{proof}

\end{enumerate}

\section{Graph}

\begin{enumerate}
  \item 
    \textbf{Theorem}. Every graph $G=(V,E)$ has at least $|V|-|E|$
    connected components. \\
    \begin{proof}
      We use induction on the number of edges. Let $P(n)$ be the
      proposition that every graph $G=(V,E)$ has at least $|V|-n$
      connected components where $|E|=n$. 

      \emph{Base case:} In a graph where $|E|=0$, every vertex is a
      connected component itself, thus the graph has exactly
      $|V|-0=|V|$ connected components. 

      \emph{Inductive step:} Assume that $P(n)$ holds for $n \geq 0$,
      that is, a graph with $|E|=n$ has at least $|V|-n$ connected
      components. Consider a graph with $n+1$ edges. Remove an edge
      $(u,v)$ to create a $n$-edge graph $G'$, which has at least
      $|V|-n$ connected components. Now add $(u,v)$ to obtain the
      original graph $G$. If
      $u$ and $v$ were in the same connected component of $G'$, then
      $G$ has the same number of connected components as $G'$. If $u$
      and $v$ were in the different connected components of $G'$, then
      adding $(u,v)$ would merge these two components of $G'$ into one in
      $G$, but all other components remain. In both cases, the number
      of connected components in $G$ is at least $|V|-n-1=|V|-(n+1)$.

      The theorem follows by induction. 
    \end{proof}
\end{enumerate}

\end{document}
