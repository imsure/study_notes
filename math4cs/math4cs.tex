%%%%%%%%%%%%%%%%%%%%%%%%%%%%%%%%%%%%%%%%%%%%%%%%%%%%%%%%%%%%%%%%%%%%%%%%%%%
% This is LaTeX file for Solutions of Homework Assignment 4
% Author: Shuo Yang
%%%%%%%%%%%%%%%%%%%%%%%%%%%%%%%%%%%%%%%%%%%%%%%%%%%%%%%%%%%%%%%%%%%%%%%%%%%

\documentclass[11pt]{article}
\usepackage{amsmath,amssymb,epsfig,graphics,hyperref,amsthm}

\hypersetup{colorlinks=true}

\setlength{\textwidth}{7in}
\setlength{\topmargin}{-0.575in}
\setlength{\textheight}{9.25in}
\setlength{\oddsidemargin}{-.25in}
\setlength{\evensidemargin}{-.25in}

\reversemarginpar
\setlength{\marginparsep}{-15mm}

\newcommand{\rmv}[1]{}
\newcommand{\bemph}[1]{{\bfseries\itshape#1}}
\newcommand{\N}{\mathbb{N}}
\newcommand{\Z}{\mathbb{Z}}
\newcommand{\imply}{\to}
\newcommand{\bic}{\leftrightarrow}

% Some user defined strings for the homework assignment
%
\def\CourseCode{Math4CS}
\def\DateHandedOut{Dec, 2014}
\def\DateDue{Dec 26}
\def\Author{Shuo Yang}

\begin{document}

\noindent

\CourseCode \hfill \DateHandedOut

\begin{center}
Induction I\\
Date: \DateDue\\
Student: \Author\\
\end{center}

% A horizontal split line
\hrule\smallskip

% Enumerate through all questions.
\begin{enumerate}

\item Courtyard Tiling\\
  The problem is to tile a courtyard with dimensions $2^n \times
  2^n$. We are required to install a statue of a wealthy donor in one
  of the central square, and only special L-shaped tiles can be
  used. We need to prove this is feasible.
  
  \textbf{Theorem} For all $n \geq 0$ there exists a tiling of a $2^n
  \times 2^n$ courtyard with the donor in a central square. 
  \begin{proof}
    Prove by induction. Let $P(n)$ be the proposition that there
    exists a tiling of a $2^n \times 2^n$ courtyard with the donor
    placed in any location.

  \emph{Base case}: $P(0)$ is true because the donor fills the whole
  courtyard.

  \emph{Inductive step}: Suppose $P(n)$ is true, we need to prove that
  $P(n) \rightarrow P(n+1)$. A $2^{n+1} \times 2^{n+1}$ courtyard
  consists of four $2^n \times 2^n$ quadrants, each of them can be tiled
  with the donor placed in any location. Let the donor be in one of
  the four central squares, and the remaining three central squares
  can fit a L-shaped tile. Now we can tile each of the four quadrants
  by the induction hypothesis. This proves that $P(n) \rightarrow
  P(n+1)$. The theorem follows as a special case.
  \end{proof}

\end{enumerate}
\end{document}
