%%%%%%%%%%%%%%%%%%%%%%%%%%%%%%%%%%%%%%%%%%%%%%%%%%%%%%%%%%%%%%%%%%%%%%%%%%%
% This is the LaTeX file for study notes for the MIT 6.042.
% Author: Shuo Yang
%%%%%%%%%%%%%%%%%%%%%%%%%%%%%%%%%%%%%%%%%%%%%%%%%%%%%%%%%%%%%%%%%%%%%%%%%%%

\documentclass[11pt]{article}
\usepackage{amsmath,amssymb,epsfig,graphics,hyperref,amsthm}

\hypersetup{colorlinks=true}

\setlength{\textwidth}{7in}
\setlength{\topmargin}{-0.575in}
\setlength{\textheight}{9.25in}
\setlength{\oddsidemargin}{-.25in}
\setlength{\evensidemargin}{-.25in}

\reversemarginpar
\setlength{\marginparsep}{-15mm}

\newcommand{\rmv}[1]{}
\newcommand{\bemph}[1]{{\bfseries\itshape#1}}
\newcommand{\N}{\mathbb{N}}
\newcommand{\Z}{\mathbb{Z}}
\newcommand{\imply}{\to}
\newcommand{\bic}{\leftrightarrow}

% Some user defined strings for the homework assignment
%
\def\CourseCode{Math4CS Notes}
\def\DateHandedOut{Dec, 2014}
\def\DateDue{Dec 28}
\def\Author{Shuo Yang}

\begin{document}

\noindent

\CourseCode \hfill \DateHandedOut

\begin{center}
Math4CS\\
Date: \DateDue\\
Student: \Author\\
\end{center}

% A horizontal split line
\hrule\smallskip

\section{Induction}

% Enumerate through all questions.
\begin{enumerate}

\item \textbf{Courtyard Tiling}\\

  The problem is to tile a courtyard with dimensions $2^n \times
  2^n$. We are required to install a statue of a wealthy donor in one
  of the central square, and only special L-shaped tiles can be
  used. We need to prove this is feasible.
  
  \textbf{Theorem} For all $n \geq 0$ there exists a tiling of a $2^n
  \times 2^n$ courtyard with the donor in a central square. 
  \begin{proof}
    Prove by induction. Let $P(n)$ be the proposition that there
    exists a tiling of a $2^n \times 2^n$ courtyard with the donor
    placed in any location.

  \emph{Base case}: $P(0)$ is true because the donor fills the whole
  courtyard.

  \emph{Inductive step}: Suppose $P(n)$ is true, we need to prove that
  $P(n) \rightarrow P(n+1)$. A $2^{n+1} \times 2^{n+1}$ courtyard
  consists of four $2^n \times 2^n$ quadrants, each of them can be tiled
  with the donor placed in any location. Let the donor be in one of
  the four central squares, and the remaining three central squares
  can fit a L-shaped tile. Now we can tile each of the four quadrants
  by the induction hypothesis. This proves that $P(n) \rightarrow
  P(n+1)$. The theorem follows as a special case.
  \end{proof}

\end{enumerate}

\section{Graph}

\begin{enumerate}
  \item 
    \textbf{Theorem}. Every graph $G=(V,E)$ has at least $|V|-|E|$
    connected components.
    \begin{proof}
      We use induction on the number of edges. Let $P(n)$ be the
      proposition that every graph $G=(V,E)$ has at least $|V|-n$
      connected components where $|E|=n$. 

      \emph{Base case:} In a graph where $|E|=0$, every vertex is a
      connected component itself, thus the graph has exactly
      $|V|-0=|V|$ connected components. 

      \emph{Inductive step:} Assume that $P(n)$ holds for $n \geq 0$,
      that is, a graph with $|E|=n$ has at least $|V|-n$ connected
      components. Consider a graph with $n+1$ edges. Remove an edge
      $(u,v)$ to create a $n$-edge graph $G'$, which has at least
      $|V|-n$ connected components. Now add $(u,v)$ to obtain the
      original graph $G$. If
      $u$ and $v$ were in the same connected component of $G'$, then
      $G$ has the same number of connected components as $G'$. If $u$
      and $v$ were in the different connected components of $G'$, then
      adding $(u,v)$ would merge these two components of $G'$ into one in
      $G$, but all other components remain. In both cases, the number
      of connected components in $G$ is at least $|V|-n-1=|V|-(n+1)$.

      The theorem follows by induction. 
    \end{proof}

  \item
    \textbf{Theorem}. Let $G$ be a digraph(possible with self-loops)
    with vertices $v_1, \cdots, v_n$. Let $M$ be the adjacency matrix
    of $G$. Then $M_{ij}^k$ is equal to the number of length-$k$ walk
    from $v_i$ to $v_j$.
    \begin{proof}
      We use induction on $k$. Let $P(k)$ be the proposition that the
      number of length-$k$ walk from $v_i$ to $v_j$ is $M_{ij}^k$, for
      all $i,j$. 

      \emph{Base case:} for $k=0$, a vertex only has length-$0$ walk
      to itself. Since $M^0$ is the identity matrix, $P(0)$ holds. 

      \emph{Inductive step:} Assume $P(k)$ holds, we will prove that
      $M_{ij}^{k+1}$ is equal to the number of length-$(k+1)$ walk
      from $v_i$ to $v_j$. A length-$(k+1)$ walk from $v_i$ to $v_j$
      consists of a walk of length-$k$ from $v_i$ to some intermediate
      vertex $v_m$ followed by an edge $(v_m, v_j)$. Therefore, the
      number of $M_{ij}^{k+1}$ is equal to:

      \begin{equation}
        M_{iv_1}^{k}M_{v_1j} + M_{iv_2}^{k}M_{v_2j} + \cdots +
        M_{iv_n}^{k}M_{v_nj} 
      \end{equation}
      This is exactly $M_{ij}^{k+1}$, thus $P(k+1)$ also holds. The
      theorem follows by induction.
    \end{proof}

  \item
    \textbf{Theorem}. Every tree $T=(V,E)$ has the following
    properties:
    \begin{enumerate}
    \item There is a unique path between every pair of vertices.
      \begin{proof}
        There is at least one path between every pair of vertices
        since the graph is connected. Suppose that there are two
        different paths between vertices $u$ and $v$. Beginning at
        $u$, let $x$ be the first vertex where paths diverge, and let
        $y$ be the next vertex they share. Then there are two paths
        between $x$ and $y$ with no common edges, which creates a
        cycle. This is a contradiction since the graph is
        acyclic. Therefore there is exactly one path between every
        pair of vertices.
      \end{proof}
    \item Adding any edge creates a cycle.
      \begin{proof}
        Suppose an edge was added between vertices $u$ and $v$. Since
        there is already a path between $u$ and $v$, this edge creates
        another path from $u$ to $v$. This contradicts the fact that
        there can be only one path between every pair of vertices.
      \end{proof}
    \item Removing any edge disconnect the graph.
      \begin{proof}
        Suppose that removing an edge between $u$ and $v$ doesn't
        disconnect the graph, then there is another path between $u$
        and $v$ exists. This contradicts the fact that
        there can be only one path between every pair of vertices.
      \end{proof}
    \item Every tree with at least two vertices has at least two
      leaves.
      \begin{proof}
        Suppose the sequence $v_1v_2 \cdots v_m$ is the longest path in
        $T$. Then $m \geq 2$, since a tree with two vertices must
        contain at least one edge. There cannot be an edge between
        $v_1$ and $v_i$ for $2 < i \leq m$, because otherwise
        $v_1 \cdots v_i$ would form a cycle. There cannot also be an
        edge between $v_1$ and $u$ for some vertex $u$ that is not on
        the path, because otherwise the path would be
        longer. Therefore the only edge incident to $v_1$ is $(v_1,
        v_2)$, which means that $v_1$ is a leaf. By a symmetric
        argument, $v_m$ is also a leaf.
      \end{proof}
    \item $|V| = |E| + 1$.
      \begin{proof}
        We use induction on $|V|$. For the tree with a single vertex,
        the claim holds because $|E|+1=0+1=1=|V|$. Suppose that the
        claim holds for all $n$-vertex tree and consider an
        $(n+1)$-vertex tree $T$. Let $v$ be a leaf of the tree. Deleting
        $v$ and its incident edge gives us a $n$-vertex tree where
        $|V|=|E|+1$ holds. Now add the vertex $v$ and its incident
        edge back, then the equation still holds because both the
        number of vertices and edges increase by 1. Thus the claim
        holds for $T$ and, by induction, for all trees.
      \end{proof}
    \end{enumerate}
\end{enumerate}

\end{document}
