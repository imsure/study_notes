%%%%%%%%%%%%%%%%%%%%%%%%%%%%%%%%%%%%%%%%%%%%%%
% Study notes on C Programming Language
% Shuo Yang
%%%%%%%%%%%%%%%%%%%%%%%%%%%%%%%%%%%%%%%%%%%%%%

\documentclass[11pt,a4paper]{article}

\title{Study notes on C Programming Language}
\author{Shuo Yang}

\begin{document}
\maketitle

\section{Two-dimensional array}

\emph{A two-dimensional array} (e.g. $A[M][N]$) actually uses a
contiguous storage structure underneath such that $A$ is the address of the first
element and $A[0]$ is the start address of the first row. We can view
it as a one-dimensional array with each row appears in sequence. So
$A=A[0]=\&A[0][0]$ and $A[1]-A[0]=N$. The type of $A$ is actually 
\emph{int $(*)[N]$} which means $A$ is a pointer to an array of size
$N$, that's how \emph{sizeof} figures out the size of $A$
correctly. The type for $A[0]$ is \emph{int *}, because it is a
pointer to a one-dimensional array. Note that a two-dimensional array
is not \emph{an array of pointers to one-dimensional array, that is,
  int *[]}, it is not efficient to access array elements as we will
have to jump back and forth to different locations.
\end{document}

