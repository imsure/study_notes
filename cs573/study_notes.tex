\documentclass[11pt]{article}
\usepackage{amsmath,amssymb,epsfig,graphics,hyperref,mdframed,
  enumitem}
\usepackage[usenames,dvipsnames,svgnames,table]{xcolor}

\definecolor{light-gray}{gray}{0.9}
\definecolor{orange}{HTML}{FF7F00}
\definecolor{light-blue}{HTML}{17F2E4}
\definecolor{yellow2}{HTML}{F8FC02}

\hypersetup{colorlinks=true}

\setlength{\textwidth}{7in}
\setlength{\topmargin}{-0.575in}
\setlength{\textheight}{9.25in}
\setlength{\oddsidemargin}{-.25in}
\setlength{\evensidemargin}{-.25in}

\newcommand{\bemph}[1]{{\bfseries\itshape#1}}

% set of space parameters for items in a list.
\setlist[enumerate]{topsep=3pt,itemsep=-1ex,partopsep=1ex,parsep=1ex}

\title{\bf Theory of Computation\\[2ex]
  \rm\normalsize Lecture delivered by Prof. Stephen Kobourov}
\author{\bf Shuo Yang}

\begin{document}

\maketitle

\begin{center}
Last updated: \date{\today}
\end{center}

\section{Introduction}

In this class, we explore into different types of \emph{computational
  model} which is an idealized computer. We start with finite automata.

\section{Regular Languages}

\emph{Finite Automata} are good models for computers with an extremely
limited amount of memory.

\subsection{DFA}

\begin{mdframed}[backgroundcolor=yellow2] % all the definitions use
                                         % yellow2 color
\textbf{Formal definition of a DFA}\\

A DFA is a 5-tuple $(Q, \Sigma, \delta, q_0, F)$, where
\begin{enumerate}
\item $Q$ is a finite set called the \emph{states},
\item $\Sigma$ is a finite set called the \emph{alphabet},
\item $\delta$: $Q \times \Sigma \rightarrow Q$
\item $q_0 \in Q$ is the \emph{transition function}
\item $F \subseteq Q$ is the \emph{set of accept states}
\end{enumerate}
\end{mdframed}

\subsection{Intuition behind the pumping lemma for regular languages}

No matter how tricky a regular language seems to be, there must be
some repetitive pattern inside the language which can be tracked by a
finite number of states.

\section{The Church-Turing Thesis}

\subsection{Turing Machine}

A \emph{Turing Machine} is a much more accurate model of a general
purpose computer. It can do everything that a real computer can do. A
TM can end up in an \emph{accept state, reject state, or an infinite loop}.

\begin{mdframed}[backgroundcolor=yellow2] % all the definitions use
                                         % yellow2 color
\textbf{Formal definition of a Turing Machine}\\

A TM is a 7-tuple $(Q, \Sigma, \Gamma, \delta, q_0, q_accept, q_reject)$, where
\begin{enumerate}
\item $Q$ is the set of states,
\item $\Sigma$ is the input alphabet not containing the blank
  symbol \textvisiblespace,
\item $\Gamma$ is the tape alphabet, where \textvisiblespace $\in 
  \Gamma$ and $\Sigma \subseteq \Gamma$,
\item $\delta: Q \times \Gamma \rightarrow Q \times \Gamma \times \{L, R\}$,
\item $q_0 \in Q$ is the start state,
\item $q_{accept} \in Q$ is the accept state, and
\item $q_{reject} \in Q$ is the reject state, and $q_{accept} \neq q_{reject}$.

\end{enumerate}
\end{mdframed}

The definition tells us that as a TM computes, changes occurs in the
current state, the current tape contents, and the current head
location. A setting of these three items is called a
\textit{\textbf{configuration}} of the Turing Machine. Say that
configuration $C_1$ \bemph{yields} configuration $C_2$ if the Turing
Machine can legally go from $C_1$ to $C_2$ in a single step.\\

Three special configurations:
\begin{enumerate}
\item \bemph{start configuration} of $M$ on input $w$ is the
  configuration $q_0w$, which indicates that the machine is in the
  start state $q_0$ with its head at the leftmost position on the
  tape.
\item \bemph{accepting configuration} the state of the configuration is
  $q_{accept}$.
\item \bemph{rejecting configuration} the state of the configuration is
  $q_{reject}$.
\end{enumerate}

Both \emph{accepting configuration} and \emph{rejecting configuration}
are \bemph{halting configurations} and do not yield further
configurations. (\emph{If a machine does not accept a string, it
  doesn't mean it rejects, it could either reject or enter an infinite
loop.})\\

The collection of strings that a Turing machine $M$ accepts is the
language of M, or the language recognized by M, denoted
$L(M)$.

\begin{mdframed}[backgroundcolor=yellow2]
\textbf{Definition}\\
Call a language \bemph{Turing-recognizable} (a.k.a \emph{recursively
  enumerable language}) if some Turing machine
recognize it. (such Turing machine may halt on inputs, or loop forever.)

\end{mdframed}

\begin{mdframed}[backgroundcolor=yellow2]
\textbf{Definition}\\
Call a language \bemph{Turing-decidable} (a.k.a \emph{recursive
  language}) or simply \bemph{decidable} if some Turing machine
decides it. (such Turing machine always halts on all inputs.)

\end{mdframed}

\begin{mdframed}[backgroundcolor=light-blue]
\textbf{Example}\\
A Turing machine that decides the language $\{w\#w| w \in \{0,1\}^*\}$.\\
$M$ = "On input string $w$:
\begin{enumerate}
\item scan from the leftmost position on the tape, cross off the first
  unmarked symbol. Then move the head all the way to the right to the
  corresponding position on the other side of the $\#$ to check
  whether these two positions contain the same symbol. If they do not,
  or if no $\#$ is found, \emph{reject}. If they do, cross off the
  matched symbol and move the head all the way back to the leftmost position.
\item repeat the previous step until all the symbols to the left of
  $\#$ have been crossed off, check for any remaining symbols to the
  right of the $\#$, if any symbols remain, \emph{reject}, otherwise,
  \emph{accept}."
\end{enumerate}

\end{mdframed}

\begin{mdframed}[backgroundcolor=light-blue]
\textbf{Example}\\
A Turing machine that decides the language $\{0^{2^n} | n \geq 0\}$.\\
$M$ = "On input string $w$:
\begin{enumerate}
\item From left to right, cross off every other $0$, if the tape contains
  a single $0$, \emph{accept}. Otherwise, if the number of $0's$ is
  odd but not $1$, \emph{reject}. 
\item Return to the left-hand end of the tape.
\item Go to stage 1."
\end{enumerate}
\end{mdframed}

\begin{mdframed}[backgroundcolor=light-blue]
\textbf{Example}\\
A Turing machine that decides the language $\{a^ib^jc^k|i \times j=k
\text{ and } i,j,k \geq 1\}$.\\
$M$ = "On input string $w$:
\begin{enumerate}
\item Scan the input from left to right to check if the input is in
  the form of $a^+b^+c^+$ and reject if it isn't.
\item Return the head to the left-hand end of the tape.
\item Cross off an $a$ and scan to the right until a $b$ is
  found. Shuttle between the $b's$ and the $c's$, crossing off one of
  each until all $b's$ are gone. If all $c's$ have been crossed off
  and some $b's$ remain, \emph{reject}.
\item Restore the crossed off $b's$ and repeat stage 3 if there is
  another $a$ to cross off. If all the $a's$ have been crossed off,
  check whether all the $c's$ also have been crossed off. If yes,
  \emph{accept}, otherwise, \emph{reject}."
\end{enumerate}
\end{mdframed}

\begin{mdframed}[backgroundcolor=light-blue]
\textbf{Example}\\
A Turing machine that decides the language\\
$\{\#x_1\#x_2\#\cdots\#x_l | \text{ each } x_i \in \{0,1\}^* \text{
  and } x_i \neq x_j \text{ for each } i \neq j \}$.\\
$M$ = "On input string $w$:
\begin{enumerate}
\item Place a mark on top of the leftmost symbol. If that symbol was a
  blank, \emph{accept}. If it was a \#, continue with the next
  stage. Otherwise, \emph{reject}.
\item Scan right to the next \# and place a second mark on top of
  it. If no \# is found before a blank symbol, only $x_1$ was present,
  \emph{accept}. Otherwise, continue with the next stage.
\item By zigzagging, compare the two strings to the right of the marked
  \#s. If they are equal, \emph{reject}. Otherwise, continue with the
  next stage.
\item Move rightmost of the two marks to the next \# to the right. If
  no \# is encountered before a blank symbol, move the leftmost mark to
  the next \# to the right and rightmost mark to the \# after that. If
  no \# is available for the rightmost mark, all the strings have been
  compared, so \emph{accept}.
\item Go to stage 3."
\end{enumerate}
\end{mdframed}

\subsection{Variants of Turing Machine}

The original Turing machine model and its reasonable variants all have
the same power - \bemph{they recognize the same class of
  languages}. This invariance to certain changes is called
\bemph{robustness}. Both FAs and PDAs are somewhat robust models, but
TMs have an as astonishing degree of robustness. The idea of proving
two TMs are equivalent in power is simply to show that we can simulate
one by another.

\section{Midterm Summary}
\subsection{RL, CFL}
$\{w|\text{w has an equal number of 0s and 1s}\}$ is not a regular
language, but is a CFL.\\
$\{1^{n^2}\ | n \geq 0\}$ is not a regular language. Is it CFL? No. Is
it context sensitive language (CSL)?\\
$\{0^n1^n\ | n \geq 0\}$ is both CFL and DCFL.\\
$\{a^ib^jc^k\ | i,j,k \geq 0 \text{ and }i=j \text{ or } i=k
\}$ and $\{ww^R | w \in \{0,1\}^*\}$ are CFLs, but not DCFLs. \\
$\{ww | w \in \{0,1\}^*\}$ is not CFL. \\
$\{a^ib^jc^k\ | 0 \leq i \leq j \leq k\}$ is not a CFL. \\
$\{a^nb^nc^n\ | 0 \leq n\}$ is not a CFL, but it is a CSL. \\
$\{1^{p}\ | p \text{ is a prime number}\}$ is not a CFL, but is a CSL.\\

The class of DCFLs is closed under complement, but \textbf{not} closed
under union, concatenation and Kleene-star.\\
If a language has a DPDA to recognize it, then it has an unambiguous
grammar. \\
The other way doesn't work! For example, $\{ww^R | w \in \{0,1\}^*\}$
has unambiguous CFG, but it is not a DCFL.

\subsection{Decidability}


\end{document}
