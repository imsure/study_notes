%%%%%%%%%%%%%%%%%%%%%%%%%%%%%%%%%%%%%%%%%%%%%%%%%%%%%%%%%%%%%%%%%%%%%%%%%%%
% This is LaTeX file for Homework Assignment 4
% Author: Shuo Yang
%%%%%%%%%%%%%%%%%%%%%%%%%%%%%%%%%%%%%%%%%%%%%%%%%%%%%%%%%%%%%%%%%%%%%%%%%%%

\documentclass[11pt]{article}
\usepackage{amsmath,amssymb,epsfig,graphics,hyperref,amsthm,mathtools}
\DeclarePairedDelimiter\ceil{\lceil}{\rceil}
\DeclarePairedDelimiter\floor{\lfloor}{\rfloor}

\hypersetup{colorlinks=true}

\setlength{\textwidth}{7in}
\setlength{\topmargin}{-0.575in}
\setlength{\textheight}{9.25in}
\setlength{\oddsidemargin}{-.25in}
\setlength{\evensidemargin}{-.25in}

\reversemarginpar
\setlength{\marginparsep}{-15mm}

\newcommand{\rmv}[1]{}
\newcommand{\bemph}[1]{{\bfseries\itshape#1}}
\newcommand{\N}{\mathbb{N}}
\newcommand{\Z}{\mathbb{Z}}
\newcommand{\imply}{\to}
\newcommand{\bic}{\leftrightarrow}

% Some user defined strings for the homework assignment
%
\def\CourseCode{CS545}
\def\AssignmentNo{4}
\def\DateHandedOut{Spring, 2015}
\def\DateDue{April 16}
\def\Author{Shuo Yang}

\begin{document}

\noindent

\CourseCode \hfill \DateHandedOut

\begin{center}
Homework Assignment \#\AssignmentNo\\
Due: \DateDue\\
Student: \Author\\
\end{center}

% A horizontal split line
\hrule\smallskip

% Enumerate through all questions.
\begin{enumerate}

\item % Problem 1

Let two stacks be $S_{rear}$ and $S_{front}$ where $S_{rear}$ is used
for putting elements to the rear of queue $Q$ and $S_{front}$ is used for
removing elements on the front of queue $Q$. To implement $Put(x, Q)$, we
just push $x$ to $S_{rear}$ such that the tail of queue would be on
top of $S_{rear}$. To implement $Get(Q)$, we pop element
from $S_{front}$, if $S_{front}$ is empty but $S_{rear}$ is not empty,
we first pop every element off $S_{rear}$ and push them to
$S_{front}$ such that the front of queue would be on top of the
$S_{front}$, then do the pop. 

\underline{Pseudo code}

\textbf{Function} $Put(x,Q)$\\
\-\hspace{2em} $push(x, S_{rear})$\\

\textbf{Function} $Get(Q)$\\
\-\hspace{2em} \textbf{if} $S_{front}$ is not empty\\
\-\hspace{4em} \textbf{return} $pop(S_{front})$\\
\-\hspace{2em} \textbf{else if} $S_{rear}$ is not empty\\
\-\hspace{4em} \textbf{while} $S_{rear}$ is not empty\\
\-\hspace{6em} $x = pop(S_{rear})$\\
\-\hspace{6em} $push(x, S_{front})$\\
\-\hspace{4em} \textbf{return} $pop(S_{front})$\\
\-\hspace{2em} \textbf{else}\\
\-\hspace{4em} \textbf{print} ``Empty Queue''\\

\underline{Amortized Analysis}

We will use the number of basic push and pop to measure cost.

For each $i=1,2,\cdots,n$, let $a_i$ be the amortized cost of the
$i$th operation, $t_i$ be the actual cost for $i$th operation, and
$D_i$ be the data structure that results after applying the $i$th
operation to data structure $D_{i-1}$. We start with $D_0$.

Let the number of elements in the stack $S_{rear}$ be $s$. 
We define the potential function $\Phi$ be $2s$. For the empty queue
$D_0$ with which we start, we have $\Phi(D_0) = 0$. Since the number
of elements in the stack is never negative, the queue $D_i$ that
results after the $i_{th}$ operation has non-negative potential, thus,
\begin{align}
  \Phi(D_i) &\geq 0\\
  &= \Phi(D_0)
\end{align}
The total amortized cost of $n$ operations with respect to $\Phi$
therefore represents an upper bound on the actual cost. 

Suppose the $i$th operation on a queue with $s$ elements in the stack
$S_{rear}$ is $Put$, then the amortized cost is:
\begin{align}
  a_i &= t_i + \Phi(D_i) - \Phi(D_{i-1})\\
  &= 1 + 2(s+1) - 2s\\
  &= 1 + 2\\
  &= 3
\end{align}
$t_i$ is 1 because $Put$ only took 1 basic push. Potential before the
operation is $2s$ and potential after the operation is $2(s+1)$ since
the size of the stack $S_{rear}$ grows by 1. Thus the change of
potential is 2.

If it is a $Get$ operation, there are two cases to consider:
\begin{enumerate}
\item stack $S_{front}$ is not empty, then the amortized cost is:
  \begin{align}
    a_i &= t_i + \Phi(D_i) - \Phi(D_{i-1})\\
    &= 1 + 2s - 2s\\
    &= 1 + 0\\
    &= 1
  \end{align}
  Again, $Get$ in this case only took 1 basic pop, so $t_i$ is 1. The
  potential didn't change since the size of stack $S_{rear}$ didn't
  change. 
\item stack $S_{front}$ is empty, then the amortized cost is:
  \begin{align}
    a_i &= t_i + \Phi(D_i) - \Phi(D_{i-1})\\
    &= (s+s+1) + 0 - 2s\\
    &= 2s + 1 - 2s\\
    &= 1
  \end{align}
  In this case, $Get$ operation took $s$ basic pop and $s$ basic push
  to remove all elements in $S_{rear}$ into $S_{front}$, and 1 basic
  pop to get the element on the front of queue. Since after the
  operation, $S_{rear}$ would be empty, thus the change of potential
  is $-2s$.
\end{enumerate}

The amortized cost for each of the two operations is $O(1)$, and thus
of total cost of a sequence of $n$ operations is $O(n)$. Since we've
already shown that the total amortized cost of $n$ operations is an
upper bound on the total actual cost. The worst-case cost of $n$
operations is therefore $O(n)$. 

\item % problem 2
  We will use an unsorted array $A$ to implement these two
  operations. Let $n$ be the size of $A$. Initially, $n=0$.

\underline{Pseudo code}

\textbf{Function} $Insert(x,S)$\\
\-\hspace{2em} $n := n + 1$\\
\-\hspace{2em} $A[n] := x$\\

\textbf{Function} $DeleteLargerHalf(S)$\\
\-\hspace{2em} use the worst-case linear time selection algorithm to
find the median of $A$.\\
\-\hspace{2em} partition the array $A$ around the median.\\
\-\hspace{2em} remove the elements from the larger half of the
partitioned array $A$.\\
\-\hspace{2em} $n := n - \ceil{n/2}$ // reset the size of $A$\\

$Insert$ takes constant time while $DeleteLargerHalf$ takes $O(n)$
time since finding the median takes linear time, and so do
partitioning the array and removing elements from the larger half. 

\underline{Amortized Analysis}

We will use the number of basic operations (operations that take constant amount
of time) to measure the cost such that the run time would be $\Theta$
of the number of basic operations. Since $Insert$ takes two basic
operations (incrementing $n$ and assigning $x$ to $A[n]$), its actual
cost would be 2. Since $DeleteLargerHalf$ takes $O(n)$ time, let its
actual cost be $cn$ where $c$ is some positive constant. 

The following table shows the real time and amortized time for each
operation. 

\begin{tabular}{ l l l  }
  operation & actual cost $t_i$ & amortized cost $a_i$ \\ \hline
  $Insert$ & 2 & 2+2$c$ \\
  $DeleteLargerHalf$ & $cn$ & 0
\end{tabular}

For $Insert$, we use 2 unit out of 2+2$c$ units to pay the actual cost and
store the remaining 2$c$ units as credit for each inserted element. For
$DeleteLargerHalf$, we use $c$ unit of credit stored on each element to pay for
the actual cost. This leaves $c$ unit of credit on each element after finding
the median and partitioning the array. When deleting the larger half,
we redistribute the $c$ unit of credit stored on each deleted element to the
remaining elements. Thus, there are always 2$c$ unit of credit stored on each
element so we can pay for future $DeleteLargerHalf$ operations.

Since each element in the array has 2$c$ unit of credit on it, and the
size of array is always non-negative, we have ensured that the amount
of credit is always non-negative. Thus, for any sequence of $n$
$Insert$ and $DeleteLargerHalf$ operations, the total amortized cost
is an upper bound on the total actual cost. 

\end{enumerate}
\end{document}
