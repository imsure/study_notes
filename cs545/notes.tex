\documentclass[11pt]{article}
\usepackage{amsmath,amssymb,epsfig,graphics,hyperref,amsthm,mathtools}
\DeclarePairedDelimiter\ceil{\lceil}{\rceil}
\DeclarePairedDelimiter\floor{\lfloor}{\rfloor}

\hypersetup{colorlinks=true}

\setlength{\textwidth}{7in}
\setlength{\topmargin}{-0.575in}
\setlength{\textheight}{9.25in}
\setlength{\oddsidemargin}{-.25in}
\setlength{\evensidemargin}{-.25in}

\reversemarginpar
\setlength{\marginparsep}{-15mm}

\newcommand{\rmv}[1]{}
\newcommand{\bemph}[1]{{\bfseries\itshape#1}}
\newcommand{\N}{\mathbb{N}}
\newcommand{\Z}{\mathbb{Z}}
\newcommand{\imply}{\to}
\newcommand{\bic}{\leftrightarrow}

% Some user defined strings for the homework assignment
%
\def\CourseCode{CS545 - Design and Analysis of Algorithms}
\def\DateHandedOut{Spring, 2015}
\def\Professor{John Kececioglu}
\def\Author{Shuo Yang}

\begin{document}

\noindent

\CourseCode \hfill \DateHandedOut

\begin{center}
Lecture Notes\\
Taught by: Professor \Professor\\
\Author\\
\end{center}

% A horizontal split line
\hrule\smallskip

\section{Divide and Conquer}
\subsection{Idea}
A Divide and Conquer algorithm consists of three phases:
\begin{enumerate}
\item Divide phase\\
Split the original problem into smaller instances of same problem
(usually $\Theta(1)$ subproblems) of size $\leq \alpha n$ for $\alpha
< 1$.
\item Conquer phase\\
Solve the subproblem recursively.
\item Combining phase\\
Combine solutions of subproblems to attain the solution to the
original problem.
\end{enumerate}

The key to applying this technique is to discover how a solution to
the input problem can be composed from (or how to decomposed into)
solution of subproblems of the same form. Problem decomposition
involves studying the structure of the solution. 

\subsection{Example}
\underline{\textbf{Largest Empty Rectangle}}\\

\underline{Input}: 2-D boolean array $A[1:m, 1:n]$ of 0s
and 1s.\\

\underline{Output}: Find a subarray $A[i:k, j:l]$ with only 0s of
largest area.\\

\underline{Exhaustive search(first attempt):} exhaustively search for
all possible rectangles that contains only 0s and find out the
largest. This takes $\Theta(n^3m^3)$ time to do, very inefficient.\\

\underline{\textbf{A divide and conquer approach:}}

Cut $A$ vertically into half at the middle column. Then the optimal
solution falls into three cases:
\begin{itemize}
\item Case-1: strictly in the left half;
\item Case-2: strictly in the right half;
\item Case-3: spans the middle.
\end{itemize}

We can recursively solve case-1 and case-2. For case-3, the largest
empty rectangle breaks into two pieces:
\begin{itemize}
\item Piece-a: left rectangle ends at mid-column and is optimal
  conditioned on starting and ending at its first and last rows. 
\item Piece-b: right rectangle starts at mid-column and is optimal
  conditioned on starting and ending at its first and last rows. 
\end{itemize}

For a given choice of row $i$ and $k$, we can find the optimal left
piece-a (starting at $i$, ending at $k$) by taking a min of the
column-left-run-length of 0s on rows $i \cdots
k$ ending at the mid-column. Similar for the optimal right piece-b. 

So we precompute:\\
$L[i,j] :=$ length of the longest run of 0s in row $i$ that ends at
$(i,j)$.\\
$R[i,j] :=$ length of the longest run of 0s in row $i$ that starts at
$(i,j)$.\\

The following pseudo code computes $L$.\\\\
\textbf{Function} $PrecomputeLeftRuns(A[1:m, 1:n])$\\
\-\hspace{2em} \textbf{for} $i := 1$ to $m$\\
\-\hspace{4em} $L[i, 0] := 0$\\
\-\hspace{4em} \textbf{for} $j := 1$ to $n$\\
\-\hspace{6em} \textbf{if} $A[i,j] == 0$\\
\-\hspace{8em} $L[i,j] := 1 + L[i, j-1]$\\
\-\hspace{6em} \textbf{else}\\
\-\hspace{8em} $L[i,j] := 0$\\

The following pseudo code computes $R$.\\\\
\textbf{Function} $PrecomputeLeftRuns(A[1:m, 1:n])$\\
\-\hspace{2em} \textbf{for} $i := 1$ to $m$\\
\-\hspace{4em} \textbf{if} $A[i,n] == 0$\\
\-\hspace{6em} $R[i,n] = 1$\\
\-\hspace{4em} \textbf{else}\\
\-\hspace{6em} $R[i,n] = 0$\\
\-\hspace{4em} \textbf{for} $j := n-1$ to $1$\\
\-\hspace{6em} \textbf{if} $A[i,j] == 0$\\
\-\hspace{8em} $R[i,j] := 1 + R[i, j+1]$\\
\-\hspace{6em} \textbf{else}\\
\-\hspace{8em} $R[i,j] := 0$\\

The following pseudo code computes case-3.\\\\
\textbf{Function} $SpanningRect(A,m,lo,hi,L,R)$\\
\-\hspace{2em} $mid := \floor{\frac{hi+lo}{2}}$\\
\-\hspace{2em} $S := 0$ // holds the area of the largest empty
rectangle spanning the mid column.\\
\-\hspace{2em} \textbf{for} $i := 1$ to $m$ // row start\\
\-\hspace{4em} $minL = \infty$\\
\-\hspace{4em} $minR = \infty$\\
\-\hspace{4em} \textbf{for} $k := 1$ to $m$ // row end\\
\-\hspace{6em} $minL = min(minL, L[k,mid])$\\
\-\hspace{6em} $minR = min(minR, R[k,mid])$\\
\-\hspace{6em} $S = max(S, (k-i+1) \times (minR+minL))$\\
\-\hspace{2em} \textbf{return} $S$\\

The final pseudo code is:\\

\textbf{Function} $LargestEmptyRect(A,m,lo,hi,L,R)$\\
\-\hspace{2em} \textbf{if} $lo < hi$ \\
\-\hspace{4em} $S := 0$\\
\-\hspace{4em} $mid := \floor{\frac{hi+lo}{2}}$\\
\-\hspace{4em} $S := max(S, LargestEmptyRect(A,m,lo,mid,L,R))$\\
\-\hspace{4em} $S := max(S, LargestEmptyRect(A,m,mid+1,hi,L,R))$\\
\-\hspace{4em} $S := max(S, SpanningRect(A,m,lo,hi,L,R))$\\
\-\hspace{4em} \textbf{return} $S$\\
\-\hspace{2em} \textbf{else} // lo == hi\\
\-\hspace{4em} \textbf{return} the length of longest run of 0s in that
column $A[1:m, lo]$\\

\end{document}
