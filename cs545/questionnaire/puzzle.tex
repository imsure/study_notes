\documentclass[11pt]{article}
\usepackage{amsmath,amssymb,epsfig,graphics,hyperref,amsthm,mathtools}
\DeclarePairedDelimiter\ceil{\lceil}{\rceil}
\DeclarePairedDelimiter\floor{\lfloor}{\rfloor}

\hypersetup{colorlinks=true}

\setlength{\textwidth}{7in}
\setlength{\topmargin}{-0.575in}
\setlength{\textheight}{9.25in}
\setlength{\oddsidemargin}{-.25in}
\setlength{\evensidemargin}{-.25in}

\reversemarginpar
\setlength{\marginparsep}{-15mm}

\newcommand{\rmv}[1]{}
\newcommand{\bemph}[1]{{\bfseries\itshape#1}}
\newcommand{\N}{\mathbb{N}}
\newcommand{\Z}{\mathbb{Z}}
\newcommand{\imply}{\to}
\newcommand{\bic}{\leftrightarrow}

% Some user defined strings for the homework assignment
%
\def\CourseCode{CS545}
\def\DateHandedOut{Spring, 2015}
\def\Author{Shuo Yang}

\begin{document}

\noindent

\CourseCode \hfill \DateHandedOut

\begin{center}
Course: \CourseCode\\
\DateHandedOut\\
Solution for the puzzle\\
Student: \Author\\
\end{center}

% A horizontal split line
\hrule\smallskip

\section*{\underline{Problem}}
\emph{input:}

Given $n$ algorithm students $s_1,s_2,\cdots,s_n$ ordered by their
ranks from 1 through $n$, and 
$m$ chocolate bars. The top-ranked student $s_1$, proposes a discrete
distribution of bars among the students. Each student votes ``yes'' or
``no'' for the proposal. If at least half the votes are ``yes'', the bars
are handed out. If the proposal fails, the top student is dismissed
and gets no bars, the process repeats among the remaining students
(that means, $s_2$ gets to propose first). Assuming students are making
their decisions independently, and smart enough to figure out the
optimal way to vote to maximize the number of bars they get, and are
extremely competitive, so each votes to dismiss the proposer if they
get the same number of bars whether or not the proposal wins.\\\\
\emph{output:}

The optimal distribution proposed by the top student $s_1$.

\section*{\underline{Algorithm}}

Assume that all $n$ students are ranked from 1 to $n$, highest to
lowest. First define some global variables.\\\\ 
$OptDist[1:n]$ := array that holds the optmial distribution, initially
, contains all 0s.\\
So $OptDist[i]$ is the number of bars distributed to the student with
the $i_{th}$ rank.\\
$m$ := total number of chocolate bars\\
$n$ := total number of students\\\\
Funtion $OptimalBarDistribution$ takes a $rank$ of a student
and produces a optimal distribution proposed by that student.\\\\
\textbf{Function} $OptimalBarDistribution( rank )$\\
\-\hspace{2em} \textbf{if} $rank == n$ // base case \\
\-\hspace{4em} $OptDist[rank] = m$ // assign all bars since no student
exists with a lower rank \\
\-\hspace{4em} \textbf{return}\\
\-\hspace{2em} $OptDist[rank] := m$ // initialize number of bars to
$rank_{th}$ student.\\
\-\hspace{2em} $next\_rank := rank+1$\\
\-\hspace{2em} $OptimalBarDistribution( next\_rank )$\\
\-\hspace{2em} \textbf{for} $i := next\_rank to n$\\
\-\hspace{4em} \textbf{if} $OptDist[i] > 0$\\
\-\hspace{6em} $OptDist[i] := 0$\\
\-\hspace{4em} \textbf{else} // $OptDist[i] == 0$\\
\-\hspace{6em} $OptDist[i] := 1$\\
\-\hspace{6em} $OptDist[i] := OptDist[i] - 1$\\

When the function terminates, $OptDist$ will contain the optimal bar
distribution proposed by the $rank_{th}$ student. So to get the
optimal bar distribution proposed by the top ranked student, just call
$OptimalBarDistribution(1)$.

\section*{\underline{Correctness}}


\end{document}
