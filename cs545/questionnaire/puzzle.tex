\documentclass[11pt]{article}
\usepackage{amsmath,amssymb,epsfig,graphics,hyperref,amsthm,mathtools}
\DeclarePairedDelimiter\ceil{\lceil}{\rceil}
\DeclarePairedDelimiter\floor{\lfloor}{\rfloor}

\hypersetup{colorlinks=true}

\setlength{\textwidth}{7in}
\setlength{\topmargin}{-0.575in}
\setlength{\textheight}{9.25in}
\setlength{\oddsidemargin}{-.25in}
\setlength{\evensidemargin}{-.25in}

\reversemarginpar
\setlength{\marginparsep}{-15mm}

\newcommand{\rmv}[1]{}
\newcommand{\bemph}[1]{{\bfseries\itshape#1}}
\newcommand{\N}{\mathbb{N}}
\newcommand{\Z}{\mathbb{Z}}
\newcommand{\imply}{\to}
\newcommand{\bic}{\leftrightarrow}

% Some user defined strings for the homework assignment
%
\def\CourseCode{CS545}
\def\DateHandedOut{Spring, 2015}
\def\Author{Shuo Yang}

\begin{document}

\noindent

\CourseCode \hfill \DateHandedOut

\begin{center}
Course: \CourseCode\\
\DateHandedOut\\
Solution for the puzzle\\
Student: \Author\\
\end{center}

% A horizontal split line
\hrule\smallskip

\section*{\underline{Problem}}
\emph{input:}

Given $n$ algorithm students $s_1,s_2,\cdots,s_n$ ordered by their
ranks from 1 through $n$, and 
$m$ chocolate bars. The top-ranked student $s_1$, proposes a discrete
distribution of bars among the students. Each student votes ``yes'' or
``no'' for the proposal. If at least half the votes are ``yes'', the bars
are handed out. If the proposal fails, the top student is dismissed
and gets no bars, the process repeats among the remaining students
(that means, $s_2$ gets to propose first). Assuming students are making
their decisions independently, and smart enough to figure out the
optimal way to vote to maximize the number of bars they get, and are
extremely competitive, so each votes to dismiss the proposer if they
get the same number of bars whether or not the proposal wins.\\\\
\emph{output:}

The optimal distribution proposed by the top student $s_1$.

\section*{\underline{Algorithm}}

Assume that all $n$ students are ranked from 1 to $n$, 1 is the
highest rank and $n$ is the lowest rank. First define some global variables.\\\\ 
$OptDist[1:n]$ := array that holds the optmial distribution, initially
, contains all 0s.\\
So $OptDist[i]$ is the number of bars distributed to the student with
the $i_{th}$ rank.\\
$m$ := total number of chocolate bars\\
$n$ := total number of students\\\\
Funtion $OptimalBarDistribution$ takes a $rank$ of a student
and produces an optimal distribution proposed by that student.\\\\
\textbf{Function} $OptimalBarDistribution( rank )$\\
\-\hspace{2em} \textbf{if} $rank == n$ // base case \\
\-\hspace{4em} $OptDist[rank] = m$ // assign all bars since no student
exists with a lower rank \\
\-\hspace{4em} \textbf{return}\\
\-\hspace{2em} $OptDist[rank] := m$ // initialize number of bars to
$rank_{th}$ student.\\
\-\hspace{2em} $next\_rank := rank+1$\\
\-\hspace{2em} $OptimalBarDistribution( next\_rank )$\\
\-\hspace{2em} \textbf{for} $i := next\_rank$ to $n$\\
\-\hspace{4em} \textbf{if} $OptDist[i] > 0$\\
\-\hspace{6em} $OptDist[i] := 0$\\
\-\hspace{4em} \textbf{else} // $OptDist[i] == 0$\\
\-\hspace{6em} $OptDist[i] := 1$\\
\-\hspace{6em} $OptDist[i] := OptDist[i] - 1$\\

When the function terminates, $OptDist$ will contain the optimal bar
distribution proposed by the $rank_{th}$ student. So to get the
optimal bar distribution proposed by the top ranked student, just call
$OptimalBarDistribution(1)$.

\section*{\underline{Correctness}}
First, suppose among all $n$ students, there are $k$ students
remaining (the other $n-k$ stduents dismissed). Let $P(k)$ be: ``among
all $k$ students, the top ranked (top rank is $n-k+1$) student's
optimal bar distribution is 
to distribute all $0s$ to $\ceil{\frac{k-1}{2}}$ students among other
students except herself, and to
distribute all $1s$ to $\floor{\frac{k-1}{2}}$ students among other
students except herself, and to distrbute $m-\floor{\frac{k-1}{2}}$ bars
to herself.\\\\
Now we prove that $P(k)$ is true using induction.
\begin{proof}
  \emph{Base case:} $P(1)$. When there is only 1 student left, his
  optimal distribution is to give all $m$ bars to himself because
  $k-1=0$. So $P(1)$ holds.\\\\
  \emph{Inductive step:} Suppose $P(k)$ holds, we need to show that
  $P(k+1)$ also holds. When there are $k+1$ students left, the number
  of students with lower rank than the top ranked student is
  $k$. Since $P(k)$ is true, the top ranked student will figure out
  the optimal distribution based on how bars are distributed by the student
  with the next highest rank. By the induction hypothesis, the next
  highest ranked student will distribute the bars this way: all $0s$
  to $\ceil{\frac{k-1}{2}}$ students among other students except
  himself, and all $1s$ to $\floor{\frac{k-1}{2}}$
  students among other students except himself, and
  $m-\floor{\frac{k-1}{2}}$ bars to himself. In order to win,
  the top ranked student must make at least $\ceil{\frac{k+1}{2}}$
  students vote ``yes''. Thus, to maximize the number of bars she can
  get, the top ranked student must distribute in this way: all $1s$ to
  $\ceil{\frac{k-1}{2}}$ students to whom the next highest ranked
  student distribute 0 bar to, they must vote ``yes'' to the top
  ranked student's proposal, otherwise they would end up getting
  nothing; all $0s$ to $\floor{\frac{k-1}{2}}$ students to whom the
  next highest ranked student distribute 1 bar to; and also $0$ to the
  next highest ranked student, and all the remaining bars to
  herself. Since each student always votes ``yes'' 
  to her own proposal, therefore the top ranked student will eventually get
  total $1+\ceil{\frac{k-1}{2}} = \ceil{\frac{k+1}{2}}$ ``yes'' votes,
  which are good enough for her to win the proposal. So the top ranked
  student's optimal distrition would be: all $0s$ to
  $\floor{\frac{k-1}{2}} + 1 = \floor{\frac{k+1}{2}} =
  \ceil{\frac{k}{2}}$ students; all $1s$ to $\ceil{\frac{k-1}{2}} = 
  \floor{\frac{k}{2}}$ students; and all the remaining bars to
  herself. Thus $P(k+1)$ also holds. \\\\
  By the principle of induction, $P(k)$ if true for all $n \in
  \mathbb{Z}^+$.  
\end{proof}

Although in order to ensure the top ranked student always wins, this
relation must hold:
\begin{equation}
  m - \floor{\frac{n-1}{2}} \geq 1
\end{equation}
When $n$ is odd, we can take off the floor:
\begin{equation}
\begin{split}
  m - \frac{n-1}{2} \geq 1\\
  m  \geq \frac{n+1}{2}
\end{split}
\end{equation}
When $n$ is even, $\floor{\frac{n-1}{2}} = \frac{n}{2}-1$, thus we have:
\begin{equation}
\begin{split}
  m -  (\frac{n}{2}-1)\geq 1\\
  m  \geq \frac{n}{2}
\end{split}
\end{equation}
So we can conclude that:
\begin{equation}
  m  \geq \floor{\frac{n+1}{2}}
\end{equation}

The correctness of the algorithm follows the proof of $P(k)$ as long
as $m \geq \floor{\frac{n+1}{2}}$.

\section*{\underline{Run time analysis}}
Let $T(n)$ be the run time for the input with $n$ students. The
recurrency equation is:
\begin{equation}
\begin{split}
  T(n) = T(n-1) + \Theta(n-1)\\
  T(n-1) = T(n-2) + \Theta(n-2)\\
  T(n-2) = T(n-3) + \Theta(n-3)\\
  \cdots\\
  T(2) = T(1) + \Theta(1)\\
  T(1) = \Theta(1)\\
\end{split}
\end{equation}
By repeatedly subsitituting the recurrence relation on the right hand
side, we have,
\begin{equation}
\begin{split}
  T(n) &= T(1) + \Theta(\sum_{i=1}^{n-1} i)\\
  &= \Theta(1) + \Theta(n^2)\\
  &= \Theta(n^2)\\
\end{split}
\end{equation}

Therefore the run time is $\Theta(n^2)$.

\section*{\underline{Answer to the puzzle}}
In the puzzle, it is the case when $n=5$ and $m=100$, the relationship
between $n$ and $m$ holds. So by running the algorithm on $n$ and $m$,
we get the optimal distribution proposed by the top ranked student is:\\\\
rank-1: 98 bars;\\ rank2: 0 bar;\\ rank-3: 1 bar;\\ rank-4: 0
bar;\\ rank-5: 1 bar. 

\end{document}
