%%%%%%%%%%%%%%%%%%%%%%%%%%%%%%%%%%%%%%%%%%%%%%%%%%%%%%%%%%%%%%%%%%%%%%%%%%%
% This is LaTeX file for Homework Assignment 2
% Author: Shuo Yang
%%%%%%%%%%%%%%%%%%%%%%%%%%%%%%%%%%%%%%%%%%%%%%%%%%%%%%%%%%%%%%%%%%%%%%%%%%%

\documentclass[11pt]{article}
\usepackage{amsmath,amssymb,epsfig,graphics,hyperref,amsthm,mathtools}
\DeclarePairedDelimiter\ceil{\lceil}{\rceil}
\DeclarePairedDelimiter\floor{\lfloor}{\rfloor}

\hypersetup{colorlinks=true}

\setlength{\textwidth}{7in}
\setlength{\topmargin}{-0.575in}
\setlength{\textheight}{9.25in}
\setlength{\oddsidemargin}{-.25in}
\setlength{\evensidemargin}{-.25in}

\reversemarginpar
\setlength{\marginparsep}{-15mm}

\newcommand{\rmv}[1]{}
\newcommand{\bemph}[1]{{\bfseries\itshape#1}}
\newcommand{\N}{\mathbb{N}}
\newcommand{\Z}{\mathbb{Z}}
\newcommand{\imply}{\to}
\newcommand{\bic}{\leftrightarrow}

% Some user defined strings for the homework assignment
%
\def\CourseCode{CS545}
\def\AssignmentNo{2}
\def\DateHandedOut{Spring, 2015}
\def\DateDue{Feb 26}
\def\Author{Shuo Yang}

\begin{document}

\noindent

\CourseCode \hfill \DateHandedOut

\begin{center}
Homework Assignment \#\AssignmentNo\\
Due: \DateDue\\
Student: \Author\\
\end{center}

% A horizontal split line
\hrule\smallskip

% Enumerate through all questions.
\begin{enumerate}

\item % Problem 1
  \underline{\textbf{Algorithm}}

  \textbf{Function} $FindClosePair(A[1:n])$\\
  \-\hspace{2em} $distance := ComputeCloseDistance(A[1:n])$\\
  \-\hspace{2em} $(x,y) := ComputeClosePair(A[1:n], distance)$\\

  \textbf{Function} $ComputeCloseDistance(A[1:n])$\\
  \-\hspace{2em} Linearly scan the input array $A[1:n]$ from left to
  right, find the minimum and maximum numbers, put them into variables
  $min$ and $max$ respectively, and let\\ 
  \-\hspace{2em} $distance := \floor{(max-min)/(n-1)}$.\\ 

  \textbf{Function} $ComputeClosePair(A[1:n], distance)$\\
  \-\hspace{2em} (1) Linearly scan the input array $A$ from left
  to right, find the minimum and maximum numbers, put them into
  variables $min$ and $max$ respectively.\\
  \-\hspace{2em} (2) Divide the $n$ elements into $\floor{n/5}$ groups of $5$ elements,
  and $\leq 1$ group of $<5$ elements.\\
  \-\hspace{2em} (3) Find the median of each group using selection sort.\\
  \-\hspace{2em} (4) Recursively find the median $med$ of the $\floor{n/5}$ medians
  found in step (3).\\
  \-\hspace{2em} (5) Partition the input array $A$ around $med$ from
  step (4) into two subarrays $A_{low}$ and $A_{high}$ such
  that $A_{low}$ contains all elements $\leq med$ and $A_{high}$ contains
  all elements $\geq med$.\\ 
  \-\hspace{2em} (6) \textbf{if} $med-min \leq distance$\\
  \-\hspace{5.5em} \textbf{return} $(med, min)$\\
  \-\hspace{3.75em} \textbf{if} $max-med \leq distance$\\
  \-\hspace{5.5em} \textbf{return} $(max, med)$\\
  \-\hspace{3.75em} \textbf{if} $n$ is odd\\
  \-\hspace{5.75em} \textbf{if} $max-med \geq med-min$\\
  \-\hspace{7.5em} $(x,y) := ComputeClosePair(A_{low}, distance)$\\
  \-\hspace{5.75em} \textbf{else} \\
  \-\hspace{7.5em} $(x,y) := ComputeClosePair(A_{high},
  distance)$\\
  \-\hspace{3.75em} \textbf{else} // $n$ is even\\
  \-\hspace{5.75em} \textbf{if} $max-med-distance \geq med-min$ // med is lower
  median \\
  \-\hspace{7.5em} $(x,y) := ComputeClosePair(A_{low}, distance)$\\
  \-\hspace{5.75em} \textbf{else} \\
  \-\hspace{7.5em} $(x,y) := ComputeClosePair(A_{high}, distance)$\\

  \underline{\textbf{Run time analysis}}\\
  The sub-routine $ComputeCloseDistance$ takes $\Theta(n)$ time. 
  Let the total run time for $ComputeClosePair$ be $T(n)$ where $n$ is
  the number of elements in the input array $A$.
  Step (1) takes $\Theta(n)$ time, step (2) takes
  $\Theta(n)$ time, step (3) takes $\Theta(n)$ time, step (4) takes
  $T(\floor{(n/5)})$ time, step (5) takes $\Theta(n)$ time, step (6)
  takes $T(n/2)$ time because for each recursive call, we reduce the
  size of the input for the sub-problem into half, that is, either
  recurse on lower partition or higher partition. Thus, we have:

  \begin{equation}
    \begin{split}
      T(n) &= T(n/5) + T(n/2) + \Theta(n) \\
      & = \Theta(n)
    \end{split}
  \end{equation}

  So the total run time for $FindClosePair$ is $T(n) + \Theta(n) =
  \Theta(n)$. 

  \underline{\textbf{Correctness}}\\
  \textbf{Lemma1}: Given an unsorted array $A$ of $n$ distinct numbers, a
  close pair always exists.

  \begin{proof}
    Prove by contradiction, that is, no such a close pair exists, this
    means for any pair $(x,y)$ where $x > y$, we have:
    \begin{equation}
      x - y > \frac{1}{n-1} (max - min)
    \end{equation}
    Suppose we order numbers in the array $A$ in ascending order as a
    sequence: $a_1, a_2, \cdots, a_n$ where $a_i > a_j$ if $i >
    j$. Let $distance = \frac{1}{n-1} (max - min)$, according to the
    assumption, we have:

    \begin{equation}
      \begin{split}
      a_n - a_{n-1} > distance\\
      a_{n-1} - a_{n-2} > distance\\
      \cdots\\
      a_3 - a_2 > distance\\
      a_2 - a_1 > distance\\
      \end{split}
    \end{equation}

    Summing the above equations together produces:
    \begin{equation}
      \begin{split}
        a_n - a_1 &> distance \times (n-1)\\
        & = \frac{1}{n-1} (max - min) \times (n-1)\\
        & = max - min\\
      \end{split}
    \end{equation}
    Since we have sorted the array $A$ in ascending order, this means
    that $a_n = max$ and $a_1 = min$, therefore the above equation says that
    $max - min > max - min$, this is clearly a contradiction, thus the lemma1
    must be true.
  \end{proof}

  \textbf{Lemma2}: Given an unsorted array $A$ of $n$ distinct
  numbers partitioned around its median into two subarrys $A_{low}$ and
  $A_{high}$. $A_{low}$ contains all elements $\leq$ median and
  $A_{high}$ contains all elements $\geq$ median. 
  Let the maximum element be $max$, minimum element be $min$ and median
  be $med$. A close pair must exist in $A_{low}$ if $max-med \geq
  med-min$ (when $n$ is odd) or $max-med-distance \geq med-min$ (when $n$ is
  even), and in $A_{high}$ otherwise. 

  \begin{proof}
    Prove by contradition.
    If $max-med \geq med-min$ (when $n$ is odd) or $max-med-distance \geq
    med-min$ (when $n$ is even), assume that $A_{low}$ does not contain
    any close pairs. There are two cases to consider:
    \begin{enumerate}
    \item $n$ is odd.\\
      In this case, $A_{low}$ contains $\frac{n-1}{2}+1=\frac{n+1}{2}$ elements,
      including the $med$ itself. Applying the same method used in
      proving Lemma1, we sort $A_{low}$ as: $a_0,
      a_1, \cdots, a_{(n+1)/2}$ in ascending order. According to the
      assumption, we must have,
      \begin{equation}
        \begin{split}
          a_{\frac{n+1}{2}} - a_{\frac{n+1}{2}-1} > distance\\
          \cdots\\
          a_3 - a_2 > distance\\
          a_2 - a_1 > distance\\
        \end{split}
      \end{equation}

      Summing the above equations together produces:
      \begin{equation}
        \begin{split}
          a_{\frac{n+1}{2}} - a_1 &> distance \times (\frac{n+1}{2}-1)\\
          & = \frac{1}{n-1} (max - min) \times (\frac{n+1}{2}-1)\\
          & = \frac{1}{n-1} (max - min) \times \frac{n-1}{2}\\
          & = \frac{max - min}{2}\\
        \end{split}
      \end{equation}

      Since $A_{low}$ is sorted in ascending order, $a_{\frac{n+1}{2}}
      = med$ and $a_1 = min$. Thus,

      \begin{equation}
        \begin{split}
          med - min > \frac{max - min}{2}\\
          med > \frac{max+min}{2}
        \end{split}
      \end{equation}

      And because $max-med \geq med-min$, we
      have $med \leq \frac{max+min}{2}$. But we have just proved that
      $med > \frac{max+min}{2}$, clearly it is a contradiction. Thus,
      there must exist a close pair in $A_{low}$.

    \item $n$ is even.\\
      In this case, there are two medians, left median and right
      median. Assume that we pick the low median.
      $A_{low}$ contains $\frac{n}{2}$ elements,
      including the $med$ itself. Applying the same method used in
      proving Lemma1, we sort $A_{low}$ as: $a_1,
      a_2, \cdots, a_{n/2}$ in ascending order. According to the
      assumption, we must have,
      \begin{equation}
        \begin{split}
          a_{\frac{n}{2}} - a_{\frac{n}{2}-1} > distance\\
          \cdots\\
          a_3 - a_2 > distance\\
          a_2 - a_1 > distance\\
        \end{split}
      \end{equation}

      Summing the above equations together produces:
      \begin{equation}
        \begin{split}
          a_{\frac{n}{2}} - a_1 &> distance \times (\frac{n}{2}-1)\\
          & = \frac{1}{n-1} (max - min) \times (\frac{n}{2}-1)\\
          & = \frac{max - min}{2} \times \frac{n-2}{n-1}\\
        \end{split}
      \end{equation}

      Since $A_{low}$ is sorted in ascending order, $a_{\frac{n}{2}}
      = med$ and $a_1 = min$, we have:

      \begin{equation}
        \begin{split}
          med - min > \frac{max - min}{2} \times \frac{n-2}{n-1}\\
          med > \frac{max+min}{2} - \frac{max-min}{2(n-1)}
        \end{split}
      \end{equation}

      Substituting $\frac{max-min}{n-1}$ with $distance$, we have:
      \begin{equation}
          med > \frac{max+min}{2} - \frac{distance}{2}
      \end{equation}

      And because $max-med-distance \geq med-min$, we
      have $med \leq \frac{max+min}{2} - \frac{distance}{2}$. But we
      have just proved that 
      $med > \frac{max-min}{2} - \frac{distance}{2}$, clearly it is a
      contradiction. Thus, there must exist a close pair in $A_{low}$.      
    \end{enumerate}
      The proof for the case when $max-med < med-min$ (when $n$ is odd) or
      $max-med-distance < med-min$ (when $n$ is even) is symmetrical.
  \end{proof}

  So combining Lemma1 and Lemma2, we can conclude that our algorithm
  can always find a pair of close elements.

\item % problem 2


\item % problem 3
  \underline{\textbf{Algorithm}}  

  \textbf{Function} $FindSmallestInMerge(A[1:m], B[1:n], k)$\\
  \-\hspace{2em} $medA := \floor{m/2}$ // index of median of A \\
  \-\hspace{2em} $medB := \floor{n/2}$ // index of median of B \\
  \-\hspace{2em} $medM := medA + medB$ // index of median of merge of A
  and B \\
  \-\hspace{2em} \textbf{if} $A[medA] > B[medB]$\\
  \-\hspace{4em} \textbf{if} $k == medM$\\
  \-\hspace{6em} \textbf{return} $A[medA]$\\
  \-\hspace{4em} \textbf{else if} $k < medM$\\
  \-\hspace{6em} $ x := FindSmallestInMerge(A[1:medA-1], B[1:medB], k)$\\ 
  \-\hspace{4em} \textbf{else} // $k > medM$\\
  \-\hspace{6em} $ x := FindSmallestInMerge(A[medA+1:m], B[medB+1:n], k-medM)$\\ 
  \-\hspace{2em} \textbf{else} // $A[medA] < B[medB]$\\
  \-\hspace{4em} \textbf{if} $k == medM$\\
  \-\hspace{6em} \textbf{return} $B[medB]$\\
  \-\hspace{4em} \textbf{else if} $k < medM$\\
  \-\hspace{6em} $ x := FindSmallestInMerge(A[1:medA], B[1:medB-1], k)$\\ 
  \-\hspace{4em} \textbf{else} // $k > medM$\\
  \-\hspace{6em} $ x := FindSmallestInMerge(A[medA+1:m], B[medB+1:1], k-medM)$\\ 

  \underline{\textbf{Run time analysis}}

  In each recursive call, we reduce the problem into half of its
  original size, and other operations executes in constant time, thus
  the recurrence equation is:

  \begin{equation}
    \begin{split}
      T(m+n) &= T(\frac{m+n}{2}) + \Theta(1)\\
      &= \Theta(\log {(m+n)})
    \end{split}
  \end{equation}

  \underline{\textbf{Correctness}}

  Since all numbers in $A$ and $B$ are distinct, there the median of
  $A$ is either $>$ or $<$ the median of $B$.
  If $A[medA] > B[medB]$, then every element in $B$ that is $\leq$
  $B[medB]$ appears before $A[medA]$ in the merged array. So $A[medA]$
  is also the median of the merged array. Depending on the value of
  $k$, the $k_{th}$ smallest could be just $A[medA]$, or lower side of
  the merge array which consists of $A[1:medA-1] and B[1:medB]$, or
  higher side of the merge array which consists of $A[medA+1:m],
  B[medB+1:n], k-medM)$ 

\end{enumerate}
\end{document}
