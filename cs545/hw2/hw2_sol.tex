%%%%%%%%%%%%%%%%%%%%%%%%%%%%%%%%%%%%%%%%%%%%%%%%%%%%%%%%%%%%%%%%%%%%%%%%%%%
% This is LaTeX file for Homework Assignment 2
% Author: Shuo Yang
%%%%%%%%%%%%%%%%%%%%%%%%%%%%%%%%%%%%%%%%%%%%%%%%%%%%%%%%%%%%%%%%%%%%%%%%%%%

\documentclass[11pt]{article}
\usepackage{amsmath,amssymb,epsfig,graphics,hyperref,amsthm,mathtools}
\DeclarePairedDelimiter\ceil{\lceil}{\rceil}
\DeclarePairedDelimiter\floor{\lfloor}{\rfloor}

\hypersetup{colorlinks=true}

\setlength{\textwidth}{7in}
\setlength{\topmargin}{-0.575in}
\setlength{\textheight}{9.25in}
\setlength{\oddsidemargin}{-.25in}
\setlength{\evensidemargin}{-.25in}

\reversemarginpar
\setlength{\marginparsep}{-15mm}

\newcommand{\rmv}[1]{}
\newcommand{\bemph}[1]{{\bfseries\itshape#1}}
\newcommand{\N}{\mathbb{N}}
\newcommand{\Z}{\mathbb{Z}}
\newcommand{\imply}{\to}
\newcommand{\bic}{\leftrightarrow}

% Some user defined strings for the homework assignment
%
\def\CourseCode{CS545}
\def\AssignmentNo{2}
\def\DateHandedOut{Spring, 2015}
\def\DateDue{Feb 26}
\def\Author{Shuo Yang}

\begin{document}

\noindent

\CourseCode \hfill \DateHandedOut

\begin{center}
Homework Assignment \#\AssignmentNo\\
Due: \DateDue\\
Student: \Author\\
\end{center}

% A horizontal split line
\hrule\smallskip

% Enumerate through all questions.
\begin{enumerate}

\item % Problem 1
  \underline{\textbf{Algorithm}}

  \textbf{Function} $FindClosePair(A[1:n])$\\
  \-\hspace{2em} $distance := ComputeCloseDistance(A[1:n])$\\
  \-\hspace{2em} $(x,y) := ComputeClosePair(A[1:n], distance)$\\

  \textbf{Function} $ComputeCloseDistance(A[1:n])$\\
  \-\hspace{2em} Linearly scan the input array $A[1:n]$ from left to
  right, find the minimum and maximum numbers, put them into variables
  $min$ and $max$ respectively, and let\\ 
  \-\hspace{2em} $distance := \floor{(max-min)/(n-1)}$.\\ 

  \textbf{Function} $ComputeClosePair(A[1:n], distance)$\\
  \-\hspace{2em} (1) Linearly scan the input array $A$ from left
  to right, find the minimum and maximum numbers, put them into
  variables $min$ and $max$ respectively.\\
  \-\hspace{2em} (2) Divide the $n$ elements into $\floor{n/5}$ groups of $5$ elements,
  and $\leq 1$ group of $<5$ elements.\\
  \-\hspace{2em} (3) Find the median of each group using selection sort.\\
  \-\hspace{2em} (4) Recursively find the median $med$ of the $\floor{n/5}$ medians
  found in step (3).\\
  \-\hspace{2em} (5) Partition the input array $A$ around $med$ from
  step (4) into two subarrays $A_{low}$ and $A_{high}$ such
  that $A_{low}$ contains all elements $\leq med$ and $A_{high}$ contains
  all elements $\geq med$.\\ 
  \-\hspace{2em} (6) \textbf{if} $med-min \leq distance$\\
  \-\hspace{5.5em} \textbf{return} $(med, min)$\\
  \-\hspace{3.75em} \textbf{if} $max-med \leq distance$\\
  \-\hspace{5.5em} \textbf{return} $(max, med)$\\
  \-\hspace{3.75em} \textbf{if} $n$ is odd\\
  \-\hspace{5.75em} \textbf{if} $max-med \geq med-min$\\
  \-\hspace{7.5em} $(x,y) := ComputeClosePair(A_{low}, distance)$\\
  \-\hspace{5.75em} \textbf{else} \\
  \-\hspace{7.5em} $(x,y) := ComputeClosePair(A_{high},
  distance)$\\
  \-\hspace{3.75em} \textbf{else} // $n$ is even\\
  \-\hspace{5.75em} \textbf{if} $max-med-distance \geq med-min$ // med is lower
  median \\
  \-\hspace{7.5em} $(x,y) := ComputeClosePair(A_{low}, distance)$\\
  \-\hspace{5.75em} \textbf{else} \\
  \-\hspace{7.5em} $(x,y) := ComputeClosePair(A_{high}, distance)$\\

  \underline{\textbf{Correctness}}\\
  \textbf{Lemma1}: Given an unsorted array $A$ of $n$ distinct numbers, a
  close pair always exists.

  \begin{proof}
    Prove by contradiction, that is, no such a close pair exists, this
    means for any pair $(x,y)$ where $x > y$, we have:
    \begin{equation}
      x - y > \frac{1}{n-1} (max - min)
    \end{equation}
    Suppose we order numbers in the array $A$ in ascending order as a
    sequence: $a_1, a_2, \cdots, a_n$ where $a_i > a_j$ if $i >
    j$. Let $distance = \frac{1}{n-1} (max - min)$, according to the
    assumption, we have:

    \begin{equation}
      \begin{split}
      a_n - a_{n-1} > distance\\
      a_{n-1} - a_{n-2} > distance\\
      \cdots\\
      a_3 - a_2 > distance\\
      a_2 - a_1 > distance\\
      \end{split}
    \end{equation}

    Summing the above equations together produces:
    \begin{equation}
      \begin{split}
        a_n - a_1 &> distance \times (n-1)\\
        & = \frac{1}{n-1} (max - min) \times (n-1)\\
        & = max - min\\
      \end{split}
    \end{equation}
    Since we have sorted the array $A$ in ascending order, this means
    that $a_n = max$ and $a_1 = min$, therefore the above equation says that
    $max - min > max - min$, this is clearly a contradiction, thus the lemma1
    must be true.
  \end{proof}

  \textbf{Lemma2}: Given an unsorted array $A$ of $n$ distinct
  numbers partitioned around its median into two subarrys $A_{low}$ and
  $A_{high}$. $A_{low}$ contains all elements $\leq$ median and
  $A_{high}$ contains all elements $\geq$ median. 
  Let the maximum element be $max$, minimum element be $min$ and median
  be $med$. A close pair must exist in $A_{low}$ if $max-med \geq
  med-min$ (when $n$ is odd) or $max-med-distance \geq med-min$ (when $n$ is
  even), and in $A_{high}$ otherwise. 

  \begin{proof}
    Prove by contradition.
    If $max-med \geq med-min$ (when $n$ is odd) or $max-med-distance \geq
    med-min$ (when $n$ is even), assume that $A_{low}$ does not contain
    any close pairs. There are two cases to consider:
    \begin{enumerate}
    \item $n$ is odd.\\
      In this case, $A_{low}$ contains $\frac{n-1}{2}+1=\frac{n+1}{2}$ elements,
      including the $med$ itself. Applying the same method used in
      proving Lemma1, we sort $A_{low}$ as: $a_0,
      a_1, \cdots, a_{(n+1)/2}$ in ascending order. According to the
      assumption, we must have,
      \begin{equation}
        \begin{split}
          a_{\frac{n+1}{2}} - a_{\frac{n+1}{2}-1} > distance\\
          \cdots\\
          a_3 - a_2 > distance\\
          a_2 - a_1 > distance\\
        \end{split}
      \end{equation}

      Summing the above equations together produces:
      \begin{equation}
        \begin{split}
          a_{\frac{n+1}{2}} - a_1 &> distance \times (\frac{n+1}{2}-1)\\
          & = \frac{1}{n-1} (max - min) \times (\frac{n+1}{2}-1)\\
          & = \frac{1}{n-1} (max - min) \times \frac{n-1}{2}\\
          & = \frac{max - min}{2}\\
        \end{split}
      \end{equation}

      Since $A_{low}$ is sorted in ascending order, $a_{\frac{n+1}{2}}
      = med$ and $a_1 = min$. Thus,

      \begin{equation}
        \begin{split}
          med - min > \frac{max - min}{2}\\
          med > \frac{max+min}{2}
        \end{split}
      \end{equation}

      And because $max-med \geq med-min$, we
      have $med \leq \frac{max+min}{2}$. But we have just proved that
      $med > \frac{max+min}{2}$, clearly it is a contradiction. Thus,
      there must exist a close pair in $A_{low}$.

    \item $n$ is even.\\
      In this case, there are two medians, left median and right
      median. Assume that we pick the low median.
      $A_{low}$ contains $\frac{n}{2}$ elements,
      including the $med$ itself. Applying the same method used in
      proving Lemma1, we sort $A_{low}$ as: $a_1,
      a_2, \cdots, a_{n/2}$ in ascending order. According to the
      assumption, we must have,
      \begin{equation}
        \begin{split}
          a_{\frac{n}{2}} - a_{\frac{n}{2}-1} > distance\\
          \cdots\\
          a_3 - a_2 > distance\\
          a_2 - a_1 > distance\\
        \end{split}
      \end{equation}

      Summing the above equations together produces:
      \begin{equation}
        \begin{split}
          a_{\frac{n}{2}} - a_1 &> distance \times (\frac{n}{2}-1)\\
          & = \frac{1}{n-1} (max - min) \times (\frac{n}{2}-1)\\
          & = \frac{max - min}{2} \times \frac{n-2}{n-1}\\
        \end{split}
      \end{equation}

      Since $A_{low}$ is sorted in ascending order, $a_{\frac{n}{2}}
      = med$ and $a_1 = min$, we have:

      \begin{equation}
        \begin{split}
          med - min > \frac{max - min}{2} \times \frac{n-2}{n-1}\\
          med > \frac{max+min}{2} - \frac{max-min}{2(n-1)}
        \end{split}
      \end{equation}

      Substituting $\frac{max-min}{n-1}$ with $distance$, we have:
      \begin{equation}
          med > \frac{max+min}{2} - \frac{distance}{2}
      \end{equation}

      And because $max-med-distance \geq med-min$, we
      have $med \leq \frac{max+min}{2} - \frac{distance}{2}$. But we
      have just proved that 
      $med > \frac{max-min}{2} - \frac{distance}{2}$, clearly it is a
      contradiction. Thus, there must exist a close pair in $A_{low}$.      
    \end{enumerate}
      The proof for the case when $max-med < med-min$ (when $n$ is odd) or
      $max-med-distance < med-min$ (when $n$ is even) is symmetrical.
  \end{proof}

  So combining Lemma1 and Lemma2, we can conclude that our algorithm
  can always find a pair of close elements.

  \underline{\textbf{Run time analysis}}\\
  The sub-routine $ComputeCloseDistance$ takes $\Theta(n)$ time. 
  Let the total run time for $ComputeClosePair$ be $T(n)$ where $n$ is
  the number of elements in the input array $A$.
  Step (1) takes $\Theta(n)$ time, step (2) takes
  $\Theta(n)$ time, step (3) takes $\Theta(n)$ time, step (4) takes
  $T(\floor{(n/5)})$ time, step (5) takes $\Theta(n)$ time, step (6)
  takes $T(n/2)$ time because for each recursive call, we reduce the
  size of the input for the sub-problem into half, that is, either
  recurse on lower partition or higher partition. Thus, we have:

  \begin{equation}
    \begin{split}
      T(n) &= T(n/5) + T(n/2) + \Theta(n) \\
      & = \Theta(n)
    \end{split}
  \end{equation}

  By the master theorem, the total run time for $FindClosePair$ is
  $T(n) + \Theta(n) = \Theta(n)$. 

\item % problem 2
  \underline{\textbf{Algorithm}}
  
  The algorithm works as the following:
  (1) Compute the index of the median: $m = \floor{n/2}$.\\
  (2) Find the $m_{th}$ smallest element $median$, which is also the median
  of the input array.\\
  (3) Partition the input array around $median$ into lower subarray and higher
  array. \\
  (4) recursively call on the lower sub-arrays, find it $(k/2)_{th}$
  quantiles, then output $median$, then recursively call on the lower
  sub-arrays, find it $(k/2)_{th}$ quantiles. Return when $k == 1$.

  \underline{\textbf{Correctness}}

  The algorithm first finds the median of the input array, which is
  also the median of the $k_{th}$ quantiles. Then partition the array
  around the median. So now we can solve two sub-problems recursively,
  each contains at most $(k-1)/2$ order statistics of the original
  input. Thus the recursion will eventually hit the base case where
  $k==1$ and return back the $k_{th}$ quantiles.

  \underline{\textbf{Run time analysis}}

  Step (1) takes constant time, step (2) takes $\Theta(n)$ time, step
  (3) takes $\Theta(n)$ time. The recursion has the depth of $\log {k}$.

  Thus the recurrence equation is:
  \begin{equation}
    T(n, k) = 2T(\frac{n}{2}, \frac{k}{2}) + \Theta(n)
  \end{equation}

  By the master theorem, the run time is $\Theta(n\log {k})$.

\item % problem 3
  \underline{\textbf{Algorithm}}  

  \textbf{Function} $FindSmallestInMerge(A[1:m], B[1:n], k)$\\
  \-\hspace{2em} $i := \floor{k/2}$ \\
  \-\hspace{2em} $j := \ceil{k/2}$ \\
  \-\hspace{2em} \textbf{if} $k == 1$ // base case \\
  \-\hspace{4em} \textbf{return} $min(A[1], B[1])$  \\
  \-\hspace{2em} \textbf{if} $A[i] > b[j]$ \\
  \-\hspace{4em} $ s_k := FindSmallestInMerge(A[1:i], B[j+1:n], i)$\\ 
  \-\hspace{2em} \textbf{else if} $A[i] < b[j]$ \\
  \-\hspace{4em} $ s_k := FindSmallestInMerge(A[i+1:m], B[1:j], j)$\\ 

  \underline{\textbf{Correctness}}

  If $A[i] > B[j]$, where $i := \floor{k/2}$ and $j := \ceil{k/2}$,
  then in the merged array, there can be at most $k-2$ elements that
  are $<$ $B[j]$, that is, $A[1:i-1]$ and $B[1:j-1]$. So we must have
  the $k_{th}$ smallest element $s_k > B[j]$. On the other hand, there
  are at least $k-1$ elements that are $< A[i]$ in the merged array,
  that is, $A[1:i-1]$ and $B[1:j]$, thus we have $s_k \leq A[i]$.

  The above analysis shows that $s_k$ can only appear in the subarray
  $B[j+1:n]$ or $A[1:i]$. Moreover, we have thrown out $j$ elements
  that $<$ $s_k$, thus, the problem is reduced to finding the $i_{th}$
  smallest element in the merged array of $B[j+1:n]$ and $A[1:i]$.

  In the case when $A[i] < B[j]$, the argument is symmetric.

  With the above argument, we can conclude that our algorithm can find
  the $k_{th}$ smallest element in the merge of two sorted arrays.

  \underline{\textbf{Run time analysis}}

  In each recursive call, we reduce the problem into half of its
  original size, and other operations executes in constant time, thus
  the recurrence equation is:

  \begin{equation}
    \begin{split}
      T(k) &= T(\frac{k}{2}) + \Theta(1)
    \end{split}
  \end{equation}
  By the master theorem, the run time is $\Theta(\log {k})$.

\item % problem 4

  \underline{\textbf{characterize the recursive structure of an
      optimal solution}} 

  For input string $S = s_1s_2 \cdots s_n$, there are three ways an
  longest palindromic subsequence (LPS) could start or end:\\
  \emph{Case 1:} LPS starts at $s_1$ and ends at $s_n$. In this
  situation, $s_1=s_n$. The optimal solution must be the LPS of
  substring $s_2 \cdots s_{n-1}$ prefixed with $s_1$ and postfixed
  with $s_n$. If not, prefixing $s_1$ and postfixing $s_n$ to the LCS
  of $s_2 \cdots s_{n-1}$ would yield a longer solution, which is a
  contradiction.

  \emph{Case 2:} LPS does not start at $s_1$. Then optimal solution
  must be LPS of $s_2 \cdots s_{n}$.

  \emph{Case 3:} LPS does not end at $s_n$. Then optimal solution
  must be LPS of $s_1 \cdots s_{n-1}$.

  \underline{\textbf{Derive a recurrence equation for the value of an
      optimal solution}}

  The recursive subproblem that arises is one of computing an LPS over
  a substring of the input string $S$. Thus the subproblem can be
  specified by the start and end index of the substring. 

  So let,
  \begin{equation}
    L(i, j) := \text{the length of LPS of substring } s_i \cdots s_j
  \end{equation}

  So, based on the three cases, the recurrence equation can be
  describe as:
  \begin{equation}
    L(i,j) := \begin{cases}
      max \begin{cases}
        L(i+1, j-1) + 2 \text{ \hspace{4em} // case 1} \\
        L(i+1, j) \text{ \hspace{4em} // case 2} \\
        L(i, j-1) \text{ \hspace{4em} // case 3} \\
      \end{cases}\\
      \-\hspace{6em} i \geq 1 \text{ and } j \geq 1 \text{ and } i < j \\

      1 \-\hspace{5.5em} i = j \\
      0 \-\hspace{5.5em} i > j \\
    \end{cases}
  \end{equation}

  The solution value for the original problem is $L(1, n)$.

  \underline{\textbf{Evaluate the recurrence bottom-up in a table}}

  We evaluate $L(i,j)$ in a table $L[0:n, 0:n]$.
  In general, the entry $(i,j)$ depends on the three entries $(i,
  j-1)$, $(i+1, j-1)$ and $(i+1, j)$. So we fill in the table 
  in diagonal-major order: first all the entries where $j-i=0$, then
  all the entries where $j-i=1$, etc.

  \textbf{Function} $EvaluateLPS(S, L, n)$\\
  \-\hspace{2em} // fill in values at diagonal\\
  \-\hspace{2em} \textbf{for} $k := 1$ to $n$\\
  \-\hspace{4em} $L[k,k] := 1$ \\

  \-\hspace{2em} // fill in values that rely only on diagonal values\\
  \-\hspace{2em} \textbf{for} $i := 1$ to $n-1$\\
  \-\hspace{4em} $j := i + 1$\\
  \-\hspace{4em} \textbf{if} $S[i] == S[j]$\\
  \-\hspace{6em} $L[i,j] = 2$\\
  \-\hspace{4em} \textbf{else}\\
  \-\hspace{6em} $L[i,j] = max(L[i,j-1], L[i+1,j])$\\

  \-\hspace{2em} // fill in remaining values \\
  \-\hspace{2em} \textbf{for} $k := 2$ to $n-1$\\
  \-\hspace{4em} \textbf{for} $i := 1$ to $n-k$\\
  \-\hspace{6em} $j := i + k$\\
  \-\hspace{6em} $L[i,j] = max(L[i,j-1], L[i+1,j], L[i+1,j-1])$\\

  \emph{Run time analysis:} First and second \textbf{for} loop both
  take $\Theta(n)$ time. The third loop takes $O(n^2)$ time. So the
  total run time is $O(n^2)$.

  \underline{\textbf{Recover an optimal solution from the table of
      solution values}}

  The function $RecoverLPS$ recovers a LPS of input string $S$ that
  starts at position $i$, ends at position $j$, given the precomputed
  table $L$.

  \textbf{Function} $RecoverLPS(S, L, i, j)$\\
  \-\hspace{2em} \textbf{if} $i > j$\\
  \-\hspace{4em} \textbf{return}\\
  \-\hspace{2em} \textbf{if} $i == j$\\
  \-\hspace{4em} output $S[i]$\\
  \-\hspace{4em} \textbf{return}\\
  \-\hspace{2em} \textbf{if} $S[i] == S[j]$ and $L[i,j] == L[i+1,
    j-1]+2$  // case 1\\ 
  \-\hspace{4em} $RecoverLCS(S, L, i+1, j-1)$\\
  \-\hspace{4em} output $S[i]$\\
  \-\hspace{2em} \textbf{else if} $L[i,j] == L[i+1,j]$ // case 2\\
  \-\hspace{4em} $RecoverLPS(S, L, i+1, j)$\\
  \-\hspace{2em} \textbf{else} $L[i,j] == L[i,j-1]$ // case 3\\
  \-\hspace{4em} $RecoverLPS(S, L, i, j-1)$\\

  \emph{Run time analysis:} each call increments $i$ or decrements
  $j$ by 1 and spends $\Theta(1)$ time. Since we start with $i=1$
  and $j=n$, it takes total $\Theta(n)$ time.

\item % problem 5

  \underline{\textbf{characterize the recursive structure of an
      optimal solution}} 

  Let the maximum independent set be $MIS$.
  For a tree rooted at some node $r$, then there are two cases for
  $MIS$: 
  \emph{case 1:} $r$ is in $MIS$. Then $MIS$ cannot contain the
  children of $r$. $MIS$ must be the sum of the optimal solutions
  rooted at each of the grandchildren of $r$, together with $r$.

  \emph{case 2:} $r$ is not in $MIS$. Then $MIS$ must be the sum of
  optimal solutions root at each of the children of $r$. 


  \underline{\textbf{Derive a recurrence equation for the value of an
      optimal solution}}

  The recursive subproblem that arises is one of computing an $MIS$ over
  a sub-tree of the input tree. Thus the problem can be specified
  by the node of the tree. Let,

  \begin{equation}
    L(v) := \text{total weights of the maximum independent set of the
      sub-tree rooted at } v
  \end{equation}

  Our goal is to compute $L(r)$.
  So, based on the three cases, the recurrence equation can be
  described as:
  \begin{equation}
    L(v) := \begin{cases}
      max\begin{cases}
      \omega(v) + \sum\limits_{\text{grandchildren of } v} L(u)\\
      \sum\limits_{\text{children of }v} L(u)\\
      \end{cases}\\
      \omega(v) \-\hspace{5.5em} $v$ \text{ is a leaf} \\
    \end{cases}
  \end{equation}


  \underline{\textbf{Evaluate the recurrence bottom-up in a table}}
  
  We evaluate the table $L$ in a bottom up fashion recursively give a
  sub-tree rooted at $v$. 

  \textbf{Function} EvaluateMIS(L, v) \\
  \-\hspace{2em} \textbf{if} $v.isLeaf$ // base case \\
  \-\hspace{4em} $L(v) := \omega(v)$ \\
  \-\hspace{4em} \textbf{return} $L(v)$ \\
  \-\hspace{2em} $sumChildren := 0$ \\
  \-\hspace{2em} $sumGrandchildren := 0$ \\
  \-\hspace{2em} \textbf{for each} child $u$ \textbf{in} $v.children$ \\
  \-\hspace{4em} $sumChildren := sumChildren + EvaluateMIS(L, u)$ \\
  \-\hspace{2em} \textbf{if} $v.hasGrandchildren$ \\
  \-\hspace{4em} \textbf{for each} grandchild $u$ \textbf{in}
  $v.grandchildren$ \\
  \-\hspace{6em} $sumGrandchildren := sumGrandchildren +
  EvaluateMIS(L,u)$ \\
  \-\hspace{4em} $L(v) := max(sumChildren, \omega(v) +
  sumGrandchildren)$ \\
  \-\hspace{4em} \textbf{return} $L(v)$ \\
  \-\hspace{2em} \textbf{else} // $v$ does not have grandchildren \\
  \-\hspace{4em} $L(v) := max(sumChildren, \omega(v)$) \\
  \-\hspace{4em} \textbf{return} $L(v)$ \\

  To compute the $MIS$ for the original input tree rooted at $r$, we
  just need to call $EvaluteMIS(L, r)$.

  \emph{Run time analysis:} there are total $n$ entries to be filled
  in. To fill each entry $L(v)$, the algorithm only looks at the
  children and grandchildren of $v$. So each vertex $u$ is accessed only
  for three times: when evaluating itself, evaluating its parent and
  when evaluating its grandparent. Since each vertex is evaluated only
  for a constant time, thus the total run time is $\Theta(n)$.\\\\\\\\\\\\


  \underline{\textbf{Recover an optimal solution from the table of
      solution values}}

  \textbf{Function} $RecoverMIS(L, v)$ \\
  \-\hspace{2em} $sumChildren := 0$ \\
  \-\hspace{2em} $sumGrandchildren := 0$ \\
  \-\hspace{2em} \textbf{if} $v.isLeaf$ \\
  \-\hspace{4em} output $v$ \\
  \-\hspace{4em} \textbf{return} \\
  \-\hspace{2em} \textbf{for each} child $u$ \textbf{in} $v.children$ \\ 
  \-\hspace{4em} $sumChildren := sumChildren + L(u)$ \\ 
  \-\hspace{2em} \textbf{if} $v.hasGrandchildren$ \\
  \-\hspace{4em} \textbf{for each} grandchild $u$ \textbf{in}
  $v.grandchildren$ \\
  \-\hspace{6em} $sumGrandchildren := sumGrandchildren + L(u)$ \\
  \-\hspace{4em} \textbf{if} $L(v) == \omega(v) + sumGrandchildren$ \\
  \-\hspace{6em} output $v$ \\
  \-\hspace{6em} \textbf{for each} grandchild $u$ \textbf{in} $v.grandchildren$\\
  \-\hspace{8em} $RecoverMIS(L, u)$ \\
  \-\hspace{4em} \textbf{else} \\
  \-\hspace{6em} \textbf{for each} child $u$ \textbf{in} $v.children$ \\ 
  \-\hspace{8em} $RecoverMIS(L, u)$\\
  \-\hspace{2em} \textbf{else} \\
  \-\hspace{4em} \textbf{if} $L(v) == sumChildren$ \\
  \-\hspace{6em} \textbf{for each} child $u$ \textbf{in} $v.children$ \\ 
  \-\hspace{8em} $RecoverMIS(L, u)$ \\
  \-\hspace{4em} \textbf{else} \\
  \-\hspace{6em} output $v$ \\
  \-\hspace{6em} \textbf{return} \\

  The procedure recovers a MIS of the sub-tree rooted at vertex $v$. So
  to recover a MIS for the original input tree rooted at $r$, just
  call $RecoverMIS(L, r)$. 

  \emph{Run time analysis:} similar with the analysis of the
  evaluation function, this time each table entry $L(v)$ will be
  looked up for at most three times, that is, when looking at its
  grandparent, its parent and itself. There are total $n$ entries,
  thus the run time is $\Theta(n)$.

\end{enumerate}
\end{document}
