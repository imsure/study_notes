%%%%%%%%%%%%%%%%%%%%%%%%%%%%%%%%%%%%%%%%%%%%%%%%%%%%%%%%%%%%%%%%%%%%%%%%%%%
% This is LaTeX file for Homework Assignment 3
% Author: Shuo Yang
%%%%%%%%%%%%%%%%%%%%%%%%%%%%%%%%%%%%%%%%%%%%%%%%%%%%%%%%%%%%%%%%%%%%%%%%%%%

\documentclass[11pt]{article}
\usepackage{amsmath,amssymb,epsfig,graphics,hyperref,amsthm,mathtools}
\DeclarePairedDelimiter\ceil{\lceil}{\rceil}
\DeclarePairedDelimiter\floor{\lfloor}{\rfloor}

\hypersetup{colorlinks=true}

\setlength{\textwidth}{7in}
\setlength{\topmargin}{-0.575in}
\setlength{\textheight}{9.25in}
\setlength{\oddsidemargin}{-.25in}
\setlength{\evensidemargin}{-.25in}

\reversemarginpar
\setlength{\marginparsep}{-15mm}

\newcommand{\rmv}[1]{}
\newcommand{\bemph}[1]{{\bfseries\itshape#1}}
\newcommand{\N}{\mathbb{N}}
\newcommand{\Z}{\mathbb{Z}}
\newcommand{\imply}{\to}
\newcommand{\bic}{\leftrightarrow}

% Some user defined strings for the homework assignment
%
\def\CourseCode{CS545}
\def\AssignmentNo{3}
\def\DateHandedOut{Spring, 2015}
\def\DateDue{March 26}
\def\Author{Shuo Yang}

\begin{document}

\noindent

\CourseCode \hfill \DateHandedOut

\begin{center}
Homework Assignment \#\AssignmentNo\\
Due: \DateDue\\
Student: \Author\\
\end{center}

% A horizontal split line
\hrule\smallskip

% Enumerate through all questions.
\begin{enumerate}

\item % Problem 1
% Enumerate through all subquestions.
\begin{enumerate} % nested enumeration
\item Suppose we have 6 activities, for activity $i$, $s_i$ is the
  start time, $f_i$ is the finish time and $d_i$ is the duration. 
And we sort them in order of increasing duration:\\\\
\begin{tabular}{ l | l l l l l l }
  $i$ & 1 & 2 & 3 & 4 & 5 & 6\\ \hline
  $s_i$ & 8 & 3 & 1 & 4 & 4 & 0\\
  $f_i$ & 9 & 5 & 4 & 8 & 9 & 6\\
  $d_i$ & 1 & 2 & 3 & 4 & 5 & 6\\
\end{tabular}

The greedy procedure when the activities are considered in order of
increasing duration would be: 1) select activity 1 since it is the
local best; 2) select activity 2 since it is compatible with activity
1; 3) no other activities can be selected since they are not
compatible with either activity 1 or 2. So the procedure yields the
solution $\{1,2\}$, but the optimal solution should be
$\{3,4,1\}$. Therefore the greedy procedure is not correct.

\item With the same input, now let us sort them in order of increasing
  start-time:\\

\begin{tabular}{ l | l l l l l l }
  $i$ & 1 & 2 & 3 & 4 & 5 & 6\\ \hline
  $s_i$ & 0 & 1 & 3 & 4 & 4 & 8\\
  $f_i$ & 6 & 4 & 5 & 8 & 9 & 9\\
\end{tabular}

The greedy procedure when the activities are considered in order of
increasing duration is:

\textbf{Function} $GreedyActivitySelection(S, F, n)$\\
\-\hspace{3em} $A := \{1\}$ // activity with the earliest start time\\
\-\hspace{3em} $j := 1$ // indicates the activity with greatest start time in $A$\\
\-\hspace{3em} \textbf{for} $i := 2$ to $n$ \\
\-\hspace{5em} \textbf{if} $S[i] \geq F[j]$ \\
\-\hspace{7em} $A := A \cup \{i\}$ \\
\-\hspace{7em} $j := i$ \\
\-\hspace{3em} \textbf{return} $A$ \\

Running the above procedure with the same input yields the solution
$\{1,6\}$ which is certainly not an optimal solution. Therefore the
greedy procedure is not correct.

\item Consider the following 9 activities ordered by the number of
  overlaps ($o_i$):\\

\begin{tabular}{ l | l l l l l l l l l }
  $i$ & 1 & 2 & 3 & 4 & 5 & 6 & 7 & 8 & 9\\ \hline
  $s_i$ & 4 & 0 & 7 & 1 & 2 & 3 & 5 & 6 & 6\\
  $f_i$ & 6 & 3 & 10 & 4 & 4 & 5 & 7 & 8 & 9\\
  $o_i$ & 2 & 2 & 2 & 3 & 3 & 3 & 3 & 3 & 3\\
\end{tabular}

When considering the order of increasing number of overlaps, the
greedy procedure would first pick activity 1, then pick activity 2,
and finally pick activity 3.
Thus the solution derived by the greedy procedure is $\{1,2,3\}$, but the
optimal solution should be $\{2,6,7,3\}$. Therefore the
greedy procedure is not correct.

\end{enumerate}

\item % problem 2


\item % problem 3
\item % problem 4

\end{enumerate}
\end{document}
